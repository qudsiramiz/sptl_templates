\documentclass[10pt,xcolor=table]{beamer}

% Use the preamble setting
% Beamer theme
\usetheme{sptl}
% Settings to have progress bar at the top
\useoutertheme[subsection=false]{miniframes}
\useinnertheme{circles}
%% \usefonttheme{structurebold}
\usecolortheme{beaver}

% Some change in font colour
\setbeamercolor{normal text}{fg=black, bg=white}
\setbeamercolor{altered text}{fg=black, bg=white}
\setbeamercolor{example text}{fg=black, bg=white}

% Change in the title color
\definecolor{BostonUniRed}{rgb}{0.8, 0.0, 0.0}
\setbeamercolor{frametitle}{fg=BostonUniRed, bg=BostonUniRed!10}

% Add graphics path
\usepackage{graphicx}
\graphicspath{{images/}{clips/}}

\usepackage{booktabs}
\usepackage[scale=2]{ccicons}

\usepackage{pgfplots}
\usepgfplotslibrary{dateplot}
\pgfplotsset{compat=1.18}
% Define graphics for logo
\titlegraphic{%
    \begin{figure}[b]
        \includegraphics[height=.125\textheight]{BostonUni.png} \hfill
        \includegraphics[height=.125\textheight]{images/sptl_v2.png}
    \end{figure}
  
}
% Setting title page to be centered, and add logos at the bottom
\makeatletter
\setbeamertemplate{title page}{
  \begin{minipage}[b][\paperheight]{\textwidth}
    % Centering titles    
    \centering  % <-- Center here
    \vfill%
    \ifx\inserttitle\@empty\else\usebeamertemplate*{title}\fi
    \ifx\insertsubtitle\@empty\else\usebeamertemplate*{subtitle}\fi
    \usebeamertemplate*{title separator}
    \ifx\beamer@shortauthor\@empty\else\usebeamertemplate*{author}\fi
    \ifx\insertdate\@empty\else\usebeamertemplate*{date}\fi
    \ifx\insertinstitute\@empty\else\usebeamertemplate*{institute}\fi
    \vfill
    % Inserting logo
    \ifx\inserttitlegraphic\@empty\else\inserttitlegraphic\fi    
    \vspace*{10mm}
  \end{minipage}
}
\setbeamertemplate{title}{
%  \raggedright%  % <-- Comment here
  \linespread{1.0}%
  \inserttitle%
  \par%
  \vspace*{0.5em}
}
\setbeamertemplate{subtitle}{
%  \raggedright%  % <-- Comment here
  \insertsubtitle%
  \par%
  \vspace*{0.5em}
}
\makeatother

% Manage frame numbering in beamer's appendixes
\usepackage{appendixnumberbeamer}

% URL
\usepackage{hyperref}
\hypersetup{
    colorlinks=true,
    linkcolor=gray,
    citecolor=blue,
    filecolor=magenta,      
    urlcolor=magenta,
    pdfpagemode=FullScreen,
    }

% references
\usepackage{cleveref}

% Citation
%style options = "numeric", "alphabetic", "authoryear", "authortitle", "verbose", "reading"
% Check this link for more options: "https://www.overleaf.com/learn/latex/Biblatex_citation_styles"
\usepackage[backend=bibtex,style=alphabetic]{biblatex}

% Change caption font size
\usepackage[font=scriptsize, labelfont=bf]{caption}
\usepackage{subcaption}

% Box around text
\usepackage{tcolorbox}

% For large table in the timeline
\usepackage{chngpage}


% Title and logo
\title[A shorter title]{Some Title for Presentation on some Random Research Topic}
\subtitle[test]{Some other details related to subtitle}
\date{\today}
\author[alphazeta@bu.edu]{Alpha B. Zeta}
\institute[SPTL, BU]{Center for Space Physics, Boston University}

% Add bibliography file. You can change bibliography style in preamble.tex file.
\bibliography{bibliographies/Papers}

%\setbeamertemplate{headline}{\hfill\includegraphics[width=1.5cm]{sptl_v2.png}\hspace{0.2cm}\vspace{-1cm}}

\begin{document}

    % Make the title
    {
    % Uncomment the following to dismiss the transparent background crest
    \setbeamertemplate{background}{\tikz[remember picture,overlay]\node[opacity=0.04] at (current page.center) {\includegraphics[height=0.8\paperheight,keepaspectratio]{BostonUni_Crest.eps}};}
    \maketitle
    }

% Setting all titles to be centered. Comment out the next line if you want to use the default
% setting of left aligned titles
\setbeamertemplate{frametitle}[default][center]

% Table of content
\begin{frame}[plain]{Table of Contents}
    \setbeamertemplate{section in toc}[sections numbered]
    %\tableofcontents[hideallsubsections]
    \tableofcontents[currentsubsection,sectionstyle=show,subsectionstyle=show/shaded]
\end{frame}

% Section 1
\section[Intro]{Introduction}
    \begin{frame}{A simple sample for a slide}
        \begin{itemize}
            \item Something important goes here.
            \item This is the other bullet point.
            \item can change the bullet style
        \end{itemize}
    \end{frame}

    \begin{frame}{List of objects using enumerate}
        How to itemize a list with different options:
        \begin{enumerate} %Options: [I], [(a)], 
            \item You can use [I] for numbering in upper case Roman numerals
            \item Use this [(a)] for numbering by letters
        \end{enumerate}
    
        \begin{itemize}
            \item This is the default bullet option
            \item[!] A point to exclaim something!
            \item[$\blacksquare$] Make the point fair and square.
            \item[NOTE] This entry has no bullet
            \item[] A blank label?
        \end{itemize}
    \end{frame}

    \begin{frame}
        How to show a specific text or image on only one slide?
        % On all slides
        \begin{center}
            \includegraphics[width=1.5in]{images/lexi_01.jpg}\hspace{0.25in}
        \end{center}
        % Only on the first slide
      \onslide<1>{
            \begin{align*}
                Author~et~al.~~(\textit{Journal}, 2022)\\
                \rm{This~is~only~on~slide~1}
            \end{align*}
        }

        % On slide 2 and 3 to the last slide
        \onslide<1-2>{
            \vspace{-0.2in}
            \begin{itemize}
                \item This item is only on slide 1 and 2.
            \end{itemize}
        }
        \onslide<3->{On slide 3 and above}
    \end{frame}

    \begin{frame}
        \frametitle{Unusual layout frame}
            \begin{columns}
                \begin{column}{.44\textwidth}
                    \begin{figure}[t]
                        \includegraphics[width=0.8\textwidth]{lexi_01.jpg}
                        \caption{A picture of a LEXI\footnotemark[1].}
                    \end{figure}
                \end{column}
            \begin{column}{.60\textwidth}
          Observations:
                \begin{itemize}
                    \item A picture of instrument.
                    \item It looks good.
                \end{itemize}
            \end{column}
        \end{columns}
        You can also split things into two columns and arrange them as so.
        \vspace{0.8cm}
    \end{frame}

% Section 2
\section{Methodology}
    \begin{frame}
        \frametitle{How to perform research}
        \begin{figure}
            \includegraphics[width=0.5\textwidth]{lexi_01.jpg}
            \caption{Hedgehog posing in a tiny kayak.}
        \end{figure}
        \vspace{-0.4cm}
        \begin{itemize}
            \item This is a picture of a happy hedgehog.
            \item You can even cite where this hedgehog is from.
        \end{itemize}
        \vspace{0.15cm}
    \end{frame}

    \begin{frame}{Different roation of images}
        \begin{columns}
            \begin{column}{0.5\textwidth}
                \begin{figure}
                    \includegraphics[width=0.45\textwidth]{lexi_01.jpg}
                    \caption{LEXI something}
                \end{figure}
                \begin{itemize}
                    \item An itemized list
                    \item Another item on list
                \end{itemize}
            \end{column}
            \begin{column}{0.5\textwidth}
                \begin{figure}
                    \includegraphics[width=0.45\textwidth, angle=90]{lexi_01.jpg}
                    \caption{Other LEXI image}
                \end{figure}
                \begin{itemize}
                    \item Itemized list for rotated image
                \end{itemize}
            \end{column}
            \vspace{0.25cm}
        \end{columns}
    \end{frame}

% Results
\section{Results}
    \begin{frame}{This section contain results}
        \begin{itemize}
            \item An important result is....
            \item Other results are 
        \end{itemize}  
      %\begin{figure}
      %  \includegraphics[width=0.65\textwidth]{mantis}
      %  \caption{Two mantis on a wig.}
      %\end{figure}
    \end{frame}

    % Conclusion and Further work
    \section{Conclusion and Further work}
    \begin{frame}{Remarks}
        These are some remarks about the work.
        \begin{itemize}
            \item This is the first remark.
            \item Here is another remark.
            \item This is the third remark.
        \end{itemize}
    \end{frame}


    % Asking questions
    \begin{frame}{Questions?}
        \begin{itemize}
            \item What is the meaning of life?
            \item What is the meaning of the universe?
            \item What is the meaning of everything?
        \end{itemize}
    \end{frame}
\end{document}
